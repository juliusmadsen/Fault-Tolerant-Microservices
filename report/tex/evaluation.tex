\section{Evaluation}
Fault tolerance techniques can be implemented for different
purposes. For some microservices the biggest possible thoughput is
wanted at the cost of an increased number of failure, others want
the highest possible success rate and accept low performance. For many
purposes it can be desirable to aim somewhere in between these two
extremes.
\\\\
Pretend that the fictive bank behind the implemented trading service
charges a fee for each succesful trade. Given the number of processed
request per second, $r$, and the failure rate, $f$, it is then in the
interest of the bank to customise the fault tolerance setup such that
their income from fees, $p$, are maximised. Thus, the goal is to
configure the fault tolerance parameters such that $p = max(r *
1-r)$.

\subsection{Test setup}
To be able to show the effects of the different implemented fault
tolerance techniques, the microservice architecture must fail in
various ways. One failure is deliberatly implemented to show the power
of timeouts but the current setup is too contrived to effectively
making the services fail in suitable ways for bulkheads and circuit
breakers to serve their intended purpose.
\\\\
For testing purposed it is assumed that the bank receives a
constant flow of 50 concurrent trade requests. The aim of the bank is
to process these requests in a way that optimises their profit.
\\\\
The system is tested by using the Apache HTTP server benchmarking
tool\footnote{https://github.com/Netflix/Hystrix} to generate 5000
trade request to the trading service at a concurrency level of 50.

\subsection{Timeouts}
Real world applications are unstable due to slow networks and
databases among many other things. To simulate some kind of
unstability, random sleeps are introduced to the GET method in the
account service. The implementation is shown in
Listing~\ref{lst:pythonsleeps}. This introduces a not completely
unrealistic behaviour of a real world microservice, where the response
time is generally noisy and some requests are responded to slowly.
\begin{lstlisting} [
	language=python,
	caption={On average 1 percent of the requests will sleep for 10 seconds,
          4 percent of the requests will sleep for 5 seconds and 18
          percent of the requests will sleep for 1
          second. Furthermore, all requests sleep for a random value
          between 0 and 0.1 seconds.},
	breaklines=true,
	label={lst:pythonsleeps}]
lucky = random.randint(0,100) 
if lucky == 100:
    time.sleep(10)
elif lucky > 95:
    time.sleep(5)
elif lucky > 80:
    time.sleep(1)
sleep_duration = random.randint(0,10)/100.0
\end{lstlisting}
The benchmark test was performed on the trader service with the
results displayed in Table~\ref{table:baselinetest}. The profit in the
base case is 67.51 successful requests per second and this is is the
number to be maximised.\\\\
\begin{table}[]
\centering
\caption{Baseline performance of trader service with no fault
  tolerance enabled. Success is the percentage of successfull requests
and profit/sec is derived by multiplying this with the number of
requests per second. 50\% of all requests finished in 313ms and no
requests took more than 10347ms.}
\label{table:baselinetest}
\begin{tabular}{|l|l|l|l|l|l|l|l|}
\hline
Requests/seq & Success & Profit/sec & 50\% & 75\% & 90\% & 95\% & 100\% \\ \hline
67.51 & 100\% & 67.51 & 313ms & 1239ms & 1239ms & 5097ms & 10347ms\\ \hline
\end{tabular}
\end{table}
Looking at Table~\ref{table:baselinetest} it shows that the slowest
ten percent of the requests are dramatically slower than the rest
of the requests. Based on this it seems sensible to set a first
attempt maximum timeout of 2 seconds, which should exclude somewhere
between 5 to 10 percent of the requests, adding room to throughput
more requests. The results of doing so are listed in
Table~\ref{table:timeouttest2sec} which shows 85.86 successful
requests per second yielding a profit increase of 27\% compared to the
base case.
\begin{table}[]
\centering
\caption{Performance of trader service with timeout set to 2 seconds.}
\label{table:timeouttest2sec}
\begin{tabular}{|l|l|l|l|l|l|l|l|}
\hline
Requests/seq & Success & Profit/sec & 50\% & 75\% & 90\% & 95\% & 100\% \\ \hline
89.52 & 95.8\% & 85.76 & 342ms & 445ms & 1343ms & 1499ms & 2256ms\\ \hline
\end{tabular}
\end{table}
There still seems to be room for improvement though so a timeout of 1
second is tried with the test results shown in
Table~\ref{table:timeouttest1sec}. This results show a decrease in
successful requests per second indicating that this timeout might be
too aggressive for this system.
\begin{table}[]
\centering
\caption{Performance of trader service with timeout set to 1 second.}
\label{table:timeouttest1sec}
\begin{tabular}{|l|l|l|l|l|l|l|l|}
\hline
Requests/seq & Success & Profit/sec & 50\% & 75\% & 90\% & 95\% & 100\% \\ \hline
104.34 & 81.44\% & 84.97 & 346ms & 438ms & 1087ms & 1119ms & 1456ms\\ \hline
\end{tabular}
\end{table}
Obiously it would be easy to set the perfect timeout for the
microservice architecture implemented here, as the underlying random
sleeps are known. For real uses this is not the case leaving only log
files and server statistics to be investigated.
\\\\
Note how the maximum runtime in both Table~\ref{table:timeouttest2sec}
and Table~\ref{table:timeouttest1sec} is greater than the timeout that
was being benchmarked. This is due to the fact that it is not the
trader service itself which times out but the requests from the trader
service to other services. Two requests from the trader service to the
account service can each take 0.9 seconds, thus avoiding the timeout
limits and adding 1.8 seconds to the total runtime of the trader service.

\subsection{Circuit breakers}
Circuit breakers comes in to effect in situations where a called
service gets overloaded. When that happens the circuit breaker makes
sure no more requests are passed through for a timeout period,
allowing the requested service to recover. To apply this to the
example microservice architecture would require that either the
account service or the stock service displayed symptoms of being
overloaded. The pure random sleeps does not adequately simulate this
situation, as there is no coherence between the sleep time of two
separate requests to the account service.
\\\\
This behaviour could be implemented by increasing the probability of a
long sleep after a long sleep has just occured, but it would be even
more artificial than the existing sleep mechanism and require a lot of
parameter tweaking to be useful for testing purposes.

\subsection{Bulkheads}
Bulkheads are effectively managing the number of threads concurrently
occupied by requesting other services. Testing of these would require
extensive knowledge and monitoring of the availability and use of
threads by each microservice. Such information is not easily
accessible in the high-level libraries used for this
implementation. It could be pursued further but doing so would lead
into working on implementation details that are not directly relevant
for the topics presented here.
\\\\
An alternative solution to test bulkheads would be to simulate thread
exhaustion by keeping track of the number of simultaneous requests but
this would again be very contrived and possibly not realistically
reflect the effects of implementing bulkheads in a real microservice
architecture.
