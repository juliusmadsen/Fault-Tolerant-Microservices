\section{Architecture}~\label{sec:architecture}
As a means of implementing a display of fault tolerance techniques a
minimal microservice architecture has been implemented. The
architecture is neither realistic nor complete but provides a useful
base to implement and test the discussed fault tolerance techniques.

\subsection{Overview}
The example microservice architecture consist of a basic banking
system and is comprised of three microservices:

\begin{itemize}
\item \textbf{The account service} in which account data can be
  created, updated and fetched. An account consists of an account ID,
  a balance and a list of owned stocks.
\item \textbf{The stock service} in which stock prices can be
  retrieved and updated.
\item \textbf{The trader service} in which stock trades are carried
  out. When a trade is requested the account service is contacted to
  get account data for the account in question  and the stock service
  is contacted to get the price of the stock in question. If the
  transaction is valid in terms of enough funds or stocks owned in the
  account to perform a trade, the account service is contacted once
  again to perform the actual trade by updating the account balance
  and the amount of owned stocks.
\end{itemize}

\subsection{Implementation}
Each microservice is implemented as a \textit{REST
API}\footnote{http://www.ics.uci.edu/~fielding/pubs/dissertation/rest\_arch\_style.htm}
in Python by use of the \textit{Flask
framework}\footnote{http://flask.pocoo.org/}. Data is stored in a \textit{Redis
database}\footnote{https://redis.io}. The implementation is based on
the \textit{microservices} repository by \textit{umermansoor} on
GitHub\footnote{https://github.com/umermansoor/microservices} and the
full source code for this project can be found in the the public
GitHub repository on
https://github.com/juliusmadsen/Fault-Tolerant-Microservices. 
\\\\
All communication to and between the services are performed through
HTTP GET and POST requests with JSON as the data format. The are five
implemented endpoints:

\begin{itemize}
\item \textbf{GET /account/<accountID>}
  \\\\
  Returns the account balance and a list of all stocks owned by the
  account with ID <accountID> in JSON:
  \\\\
    \textit{
    \{
    balance: <balance>,
    stocks:
    [
      \{
      stockName: <stockName>,
      amount: <amount>
      \}
    ]
    \}
  }.
\item \textbf{POST /account/<accountID>} with \emph{JSON} body
  \textit{
    \{
      amount: <balanceDiff>,
      stock:
        \{
          name: <name>,
          amount: <amount>
        \}
    \}
  }
  \\\\
  Updates the account with ID <accountID> by adding <balanceDiff> to
  the account balance and/or adding <amount> to the number of stocks with
  name <name> already owned.
  \\\\
  Returns a JSON object of the form
  \textit {
    \{
      updated: True,
      balance: <balance>,
      stock:
        \{
          <stockName>: <stockAmount>
        \}
    \}
  }
  if the input data is accepted or simply
  \textit{
    \{ updated: False \}
  }
  if the input data is rejected.
\item \textbf{GET /stock/<stock name>}
  \\\\
  Returns a JSON object of the form \textit{
    \{ quote: <quote> \}
  } where <quote> is the stock quote for the stock with name <stock name>.
\item \textbf{POST /stock/<stock name>} with \emph{JSON} body \{ price:
  <price> \}.
  \\\\
  Sets the price of <stock name> to <price>.
  \\\\
  Returns a JSON object of the form \textit{
    \{ success: <Bool> \}
  } with <Bool> being True on success and False on failure.
\item \textbf{POST /trade} with \emph{JSON} body \{ accountId: <accountId>, stockName:
  <stockName>, amount: <amount> \}.
  \\\\
  Requests a stock trade for account with ID <accountID> to buy or
  sell <amount> shares of stock with name <stockName>.
  \\\\
  Returns a JSON object of the form
  \textit {
    \{
      updated: True,
      balance: <balance>,
      stock:
        \{
          <stockName>: <stockAmount>
        \}
    \}
  }
  if successful or
    \textit {
    \{
      success: False,
      message: <message>
    \}
  }
  on failure with <message> being the reason for the failure.

\end{itemize}

The implementation of this microservice architecture has been carried
out with fault-tolerance in mind and thus only serves the purpose of
showcasing fault-tolerance techniques. No form of authentication or
security has been implemented.
