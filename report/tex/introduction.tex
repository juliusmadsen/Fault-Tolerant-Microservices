\section{Introduction}
This project deals with fault tolerant techniques in relation to
\textit{microservices}. In order to do so, \textit{microservices} as a
general concept is introduced, followed by the design and
implementation of a small microservice architecture that serves as the
basis for the rest of the project.
\\\\
Fault tolerance in microservices can be implemented directly in the
microservices themselves by use of special programming
techniques. Also, the hardware and software environment in which the
microservices operate have a big influence on their fault
tolerance. Getting the setup right can provide great fault tolerance
in terms of replication and redundancy.
\\\\
These two different paths to achieve higher fault tolerance in
microservices compliment each other nicely, and the topic for the
remaining part of the project is to both discuss them and implement
the ideas presented in to a microservice architecture.
\\\\
There exists hugely popular libraries such as \textit{Netflix Hystrix}
which provides easy to use fault tolerance techniques out of the
box. These libraries will not be discussed further, as the main aim of
this project is to investigate the inner workings of the underlying
techniques instead of only being able to use them.
