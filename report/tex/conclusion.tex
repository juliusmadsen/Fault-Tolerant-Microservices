\section{Conclusion}
Throughout this project different methods for improving fault
tolerance in microservice architectures have been investigated. A lot
of time has been adressed to implementation and practical issues in
order to gain a deeper understanding of not only the ideas presented
but also how to use them.
\\\\
Not all techniques are necessarily suitable for all microservice
architectures as the underlying reasons for errors that occur can
obviously differ from one system to another. The techniques presented
here are known best practices that can be suited for the individual
use case.
\\\\
It is evident that all of the discussed ideas can greatly enhance fault
tolerance in microservices, although not much proper testing has been
carried out to back up this claim. The best way of doing so would be
to have a real world microservice setup on which the different
techniques could be implemented and tested. Lacking one such leaves
only the option of trying to simulate a system including a realistic
work-load. This is a sizeable task which is out of scope for this
project.
